\section{Introduction}
\label{sec:introduction}

The comparison of experimental data for prompt production of \mbox{S-wave} quarkonia,
{\it{e.g.}}  \jpsi~or~\Y1S~mesons, with theory predictions requires
knowledge of feed-down contributions from P-wave quarkonia
states, \eg~radiative \decay{\chib}{\ups\g}~decays.
This contribution could significantly influence
the~interpretation of the~measured polarization of S-wave vector
quarkonia. In addition, measurements of the~relative production rates of P-wave to S-wave
quarkonia, as well as the~tensor-to-vector ratios, provide valuable
information on colour-octet matrix elements.


Depending on the~relative orientation of the~quark spins,
the~\chib~states can be either scalar, vector or tensor mesons, denoted by
$\Pchi_{\bquark \mathrm{J}}$ with total angular momentum $\mathrm{J}=0,1,2$.
%%
The~fractions of \YnS~decays
originating from \chibmp~decays,
where $\mathrm{n}$~and~$\mathrm{m}$ are radial quantum numbers of
the~bound states, are defined as
\begin{equation}
  \mathcal{R}^{\chibmp}_{\YnS} \equiv
  \dfrac
      {\upsigma  \left( \proton\proton \to \Pchi_{\bquark1}\mathrm{(mP)} \mathrm{X} \right)}
      {\upsigma  \left( \proton\proton \to \YnS    \mathrm{X} \right)}
      \times \BR_1
      + \dfrac
      {\upsigma  \left( \proton\proton \to \Pchi_{\bquark2}\mathrm{(mP)} \mathrm{X} \right)}
      {\upsigma  \left( \proton\proton \to \YnS    \mathrm{X} \right)}
      \times \BR_2
      \label{eq:r},
\end{equation}
%%
where $\BR_{1(2)}$ denotes the branching fraction 
for the~decay \decay{\Pchi_{\bquark1(2)}\mathrm{(mP)}}{\YnS \g}.
Possible contributions from \decay{\Pchi_{\bquark0}\mathrm{(mP)}}{\YnS \g}~decays 
are neglected because  of the~small branching fraction for the~corresponding
radiative decays.

A measurement of the fractions $\mathcal{R}^{\chibmp}_{\YnS}$ was
recently performed by LHCb using calorimetry photons~\cite{Aaij:1746553}.
This measurements supersede earlier \lhcb~measurements~\cite{LHCb-PAPER-2012-015,LHCb-CONF-2012-020}
by increasing the~statistical precision and exploiting more decay modes 
in the rapidity range  $2.0<y<4.5$.
