\section{Results}
\label{sec:Conclusion}


The measured fractions \Rmn~\cite{Aaij:1746553} are presented in Fig.~\ref{fig:frac}.
The results are dominated by the statistical uncertainties, and show no dependence on the \proton\proton~collision energy.
A~measurement of the~$\mathcal{R}^{\chibthreep}_{\Y3S}$~fraction is performed for the first time.
The large value of this fraction impacts the~interpretation
of experimental data on \ups~production and polarization.
When data on \ups~production and polarization are compared with theory predictions,
as well as when different theory predictions are compared among themselves, it is often implicitly assumed
that the~fraction of \Y3S~mesons produced by feed~down from higher states is small.
The~large measured value of  $\mathcal{R}^{\chibthreep}_{\Y3S}$
indicates that these assumptions need to be revisited. 


\begin{figure*}[t!]
  \setlength{\unitlength}{1mm}
  \centering
  \begin{picture}(150,120)
    %% =======================================================================
    \put(0,0){
      %%\includegraphics*[width=75mm, height=60mm]{frac_ups3s.eps}
      \includegraphics*[width=75mm, height=60mm]{Fig-c.eps}
    }
    \put(2,25){\begin{sideways}$\mathcal{R}^{\chibthreep}_{\Y3S}~~~~~~~~\left[\%\right]$ \end{sideways}}
    \put(35,0){$p_{\mathrm{T}}^{\Y3S}$}
    \put(60,0){$\left[\!\gevc\right]$}
    
    \put(55,53){\scriptsize \textcolor{blue}{\sqs=7\tev}}
    \put(55,49){\scriptsize \textcolor{red}{\sqs=8\tev}}
    
    \put(15,54){\tiny \textcolor{blue}{\ding{111}} \textcolor{red}{\ding{110}}  $\chibThreeP \to \Y3S \g$}
      
    \put(15,40){\lhcb}
    %% =======================================================================
    \put(0,60){
      %%\includegraphics*[width=75mm, height=60mm]{frac_ups1s.eps}
      \includegraphics*[width=75mm, height=60mm]{Fig-a.eps}
    }
    \put(2,85){\begin{sideways}$\mathcal{R}^{\chibmp}_{\Y1S}~~~~~~~~\left[\%\right]$ \end{sideways}}
    \put(35,60){$p_{\mathrm{T}}^{\Y1S}$}
    \put(60,60){$\left[\!\gevc\right]$}
    
    \put(55,113){\scriptsize \textcolor{blue}{\sqs=7\tev}}
    \put(55,109){\scriptsize \textcolor{red}{\sqs=8\tev}}
    
    \put(14.8,114){\tiny \textcolor{blue}{\ding{109}} \textcolor{red}{\ding{108}} $\chibOneP \to \Y1S \g$}
    
    \put(14.8,111){\scalebox{0.6}{\textcolor{blue}{$\vartriangle$} \textcolor{red}{$\blacktriangle$}}\,\tiny$\chibTwoP \to \Y1S \g$}
    
    \put(15,108){\tiny \textcolor{blue}{\ding{111}} \textcolor{red}{\ding{110}} $\chibThreeP \to \Y1S \g$}    
    
    \put(15,100){\lhcb}

    %% =======================================================================
    \put(75,60){
      %%\includegraphics*[width=75mm, height=60mm]{frac_ups2s.eps}
      \includegraphics*[width=75mm, height=60mm]{Fig-b.eps}
    }
    \put(77,85){\begin{sideways}$\mathcal{R}^{\chibmp}_{\Y2S}~~~~~~~~\left[\%\right]$ \end{sideways}}
    \put(110,60){$p_{\mathrm{T}}^{\Y2S}$}
    \put(135,60){$\left[\!\gevc\right]$}
        
    \put(130,113){\scriptsize \textcolor{blue}{\sqs=7\tev}}
    \put(130,109){\scriptsize \textcolor{red}{\sqs=8\tev}}
    
    
    \put(89.8,114){\scalebox{0.6}{\textcolor{blue}{$\vartriangle$} \textcolor{red}{$\blacktriangle$}}\,\tiny$\chibTwoP \to \Y2S \g$}
    
    \put(90,111){\tiny \textcolor{blue}{\ding{111}} \textcolor{red}{\ding{110}} $\chibThreeP \to \Y2S \g$}  

    \put(90,100){\lhcb}

    % \graphpaper[5](0,0)(150, 120)
 \end{picture}
 \caption{\small
        Fractions \Rmn as functions of $p_{\mathrm{T}}^{\ups}$~\cite{Aaij:1746553}.
        Points with blue~open\,(red solid) symbols
        correspond to data collected at $\sqs=7(8)\,\mathrm{TeV}$, respectively.
        For better visualization the~data points are slightly displaced
        from the~bin centres.
        The~inner error bars represent statistical uncertainties,
        while the~outer error bars indicate statistical and systematic uncertainties added in quadrature. }
 \label{fig:frac}
\end{figure*}


The recent results supersede earlier \lhcb~measurements~\cite{LHCb-PAPER-2012-015,LHCb-CONF-2012-020}
by increasing the~statistical precision and exploiting more decay modes and higher transverse 
momentum regions.
In particular, the full data sample collected by \lhcb at \mbox{$\sqs=7$}~and 8\tev has been used and
the~measured fractions $\mathcal{R}^{\chibmp}_{\YnS}$
are reported for all six kinematically allowed transitions:
\mbox{\decay{\chibonep}{\Y1S\g}},
\mbox{\decay{\chibtwop}{\Y1S\g}},
\mbox{\decay{\chibtwop}{\Y2S\g}},
\mbox{\decay{\chibthreep}{\Y1S\g}},
\mbox{\decay{\chibthreep}{\Y2S\g}} and 
\mbox{\decay{\chibthreep}{\Y3S\g}}
in bins of transverse momentum of the~\ups~mesons in the rapidity range  $2.0<y<4.5$.
\lhcb also measured the fractions $\mathcal{R}^{\chibmp}_{\YnS}$ with the radiative photon reconstruted from a $\epem$ pair~$\mathcal{R}^{\chibmp}_{\YnS}$~\cite{Aaij:1753641}. The results are consistent with those obtained in~\cite{Aaij:1746553}.

Using photons converted into electron pairs, and assuming the mass splitting $m_{\chibtwoThreeP}-m_{\chiboneThreeP}=10.5\mevcc$,
the~mass of the \chiboneThreeP~state is measured to be 
\begin{equation*} 
m_{\chiboneThreeP} = 10515.7^{+2.2}_{-3.9}\stat ^{+1.5}_{-2.1}\syst \mevcc,
\end{equation*}
This result \cite{Aaij:1753641} is compatible and significantly more precise than 
the event yield average mass of
\chiboneThreeP~and
\chibtwoThreeP~states,
\mbox{$10\,530 \pm 5\pm 17\mevcc$} and 
\mbox{$10\,551 \pm 14\pm 17\mevcc$}, 
reported by the~\atlas~\cite{Aad:2011ih}
and D0~\cite{Abazov:2012gh} experiments, respectively.


