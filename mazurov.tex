%====================================================================%
%                  17LOMCON.TEX     October 2015                     %
% This LaTeX file has adapted various sources for the use in the     %
%      preparation of the standard Proceedings Volume                %
%====================================================================%

\documentclass[a4paper]{article}

\usepackage{17lomcon}        % Proceedings volume layout metrics
\usepackage{ifthen} % for conditional statements
\usepackage{cite}             % Smart range citations
\usepackage{epsfig}           % Encapsulated PostScript figure inclusion
\usepackage{xspace}
\usepackage{rotating}
\usepackage{color}
\usepackage{pifont}
\usepackage{multirow}
\usepackage{lineno}
%% Math
\usepackage{amsmath} % Adds a large collection of math symbols
\usepackage{amssymb}
\usepackage{amsfonts}
\usepackage{upgreek} % Adds in support for greek letters in roman typeset
%\usepackage{epstopdf}
\graphicspath{{./figs/}} % Make Latex search fig subdir for figures

\bibliographystyle{unsrt}    % for BibTeX - sorted numerical labels by order of
                             % first citation.


%% =============================================================================
\newboolean{inbibliography}
\setboolean{inbibliography}{false} %True once you enter the bibliography
%% BEGIN lhcb symbols
\def\lhcb {\mbox{LHCb}\xspace}
\def\atlas  {\mbox{ATLAS}\xspace}
\def\cms    {\mbox{CMS}\xspace}
\def\Pgamma      {\ensuremath{\upgamma}\xspace}                 
\def\Pchi        {\ensuremath{\upchi}\xspace}                 
\def\Ppsi        {\ensuremath{\uppsi}\xspace}                 
\def\PUpsilon      {\ensuremath{\Upsilon}\xspace}                 
 

\def\PJ      {\ensuremath{\mathrm{J}}\xspace}                 
\def\Pa      {\ensuremath{\mathrm{a}}\xspace}                 
\def\Pb      {\ensuremath{\mathrm{b}}\xspace}                 
\def\Pe      {\ensuremath{\mathrm{e}}\xspace}                 
\def\Pp      {\ensuremath{\mathrm{p}}\xspace}                 

\def\epem       {{\ensuremath{\Pe^+\Pe^-}}\xspace}
\def\g      {{\ensuremath{\Pgamma}}\xspace}
\def\bquark    {{\ensuremath{\Pb}}\xspace}
%% Onia

\def\jpsi     {{\ensuremath{{\PJ\mskip -3mu/\mskip -2mu\Ppsi\mskip 2mu}}}\xspace}
\def\psitwos  {{\ensuremath{\Ppsi{\mathrm{(2S)}}}}\xspace}
\def\psiprpr  {{\ensuremath{\Ppsi(3770)}}\xspace}
\def\etac     {{\ensuremath{\Peta_\cquark}}\xspace}
\def\chiczero {{\ensuremath{\Pchi_{\cquark 0}}}\xspace}
\def\chicone  {{\ensuremath{\Pchi_{\cquark 1}}}\xspace}
\def\chictwo  {{\ensuremath{\Pchi_{\cquark 2}}}\xspace}
  %\mathchardef\Upsilon="7107
\def\Y#1S{\ensuremath{\PUpsilon\mathrm{(#1S)}}\xspace}% no space before {...}!
\def\OneS  {{\Y1S}}
\def\TwoS  {{\Y2S}}
\def\ThreeS{{\Y3S}}
\def\FourS {{\Y4S}}
\def\FiveS {{\Y5S}}

\def\chic  {{\ensuremath{\Pchi_{\cquark}}}\xspace}

\def\proton      {{\ensuremath{\Pp}}\xspace}
%% Decays
\def\BF         {{\ensuremath{\cal B}}\xspace}
\def\BR         {\BF}
\newcommand{\decay}[2]{\ensuremath{#1\!\to #2}\xspace}         % {\Pa}{\Pb \Pc}
\def\to                 {\ensuremath{\rightarrow}\xspace}
%% Energy and momentum
\newcommand{\tev}{\ifthenelse{\boolean{inbibliography}}{\ensuremath{~T\kern -0.05em eV}\xspace}{\ensuremath{\mathrm{\,Te\kern -0.1em V}}}\xspace}
\newcommand{\gev}{\ensuremath{\mathrm{\,Ge\kern -0.1em V}}\xspace}
\newcommand{\mev}{\ensuremath{\mathrm{\,Me\kern -0.1em V}}\xspace}
\newcommand{\kev}{\ensuremath{\mathrm{\,ke\kern -0.1em V}}\xspace}
\newcommand{\ev}{\ensuremath{\mathrm{\,e\kern -0.1em V}}\xspace}
\newcommand{\gevc}{\ensuremath{{\mathrm{\,Ge\kern -0.1em V\!/}c}}\xspace}
\newcommand{\mevc}{\ensuremath{{\mathrm{\,Me\kern -0.1em V\!/}c}}\xspace}
\newcommand{\gevcc}{\ensuremath{{\mathrm{\,Ge\kern -0.1em V\!/}c^2}}\xspace}
\newcommand{\gevgevcccc}{\ensuremath{{\mathrm{\,Ge\kern -0.1em V^2\!/}c^4}}\xspace}
\newcommand{\mevcc}{\ensuremath{{\mathrm{\,Me\kern -0.1em V\!/}c^2}}\xspace}


\def\invfb   {\ensuremath{\mbox{\,fb}^{-1}}\xspace}
   

\newcommand{\stat}{\ensuremath{\mathrm{\,(stat)}}\xspace}
\newcommand{\syst}{\ensuremath{\mathrm{\,(syst)}}\xspace}

%% Energy, Momenta
\def\sqs   {\ensuremath{\protect\sqrt{s}}\xspace}
\newcommand{\eg}{\mbox{\itshape e.g.}\xspace}

%% END lhcb symbols
%% BEGIN locasymbols
\def\chib           {\ensuremath{\Pchi_{\bquark}}\xspace}
\def\chibzero       {\ensuremath{\Pchi_{\bquark 0}}\xspace}
\def\chibone        {\ensuremath{\Pchi_{\bquark 1}}\xspace}
\def\chibtwo        {\ensuremath{\Pchi_{\bquark 2}}\xspace}

\def\chibOneP       {\ensuremath{\Pchi_{\bquark}\mathrm{(1P)}}\xspace}
\def\chibTwoP       {\ensuremath{\Pchi_{\bquark}\mathrm{(2P)}}\xspace}
\def\chibThreeP     {\ensuremath{\Pchi_{\bquark}\mathrm{(3P)}}\xspace}

\def\chiboneOneP    {\ensuremath{\Pchi_{\bquark 1}\mathrm{(1P)}}\xspace}
\def\chiboneTwoP    {\ensuremath{\Pchi_{\bquark 1}\mathrm{(2P)}}\xspace}
\def\chiboneThreeP  {\ensuremath{\Pchi_{\bquark 1}\mathrm{(3P)}}\xspace}

\def\chibtwoOneP    {\ensuremath{\Pchi_{\bquark 2}\mathrm{(1P)}}\xspace}
\def\chibtwoTwoP    {\ensuremath{\Pchi_{\bquark 2}\mathrm{(2P)}}\xspace}
\def\chibtwoThreeP  {\ensuremath{\Pchi_{\bquark 2}\mathrm{(3P)}}\xspace}


\def\ups            {\ensuremath{\PUpsilon}\xspace}
\def\YnS            {\ensuremath{\PUpsilon\mathrm{(nS)}}\xspace}% no space before {...}!

\def\chibonep       {\ensuremath{\Pchi_{\bquark}\mathrm{(1P)}}\xspace}
\def\chibtwop       {\ensuremath{\Pchi_{\bquark}\mathrm{(2P)}}\xspace}
\def\chibthreep     {\ensuremath{\Pchi_{\bquark}\mathrm{(3P)}}\xspace}
\def\chibmp         {\ensuremath{\Pchi_{\bquark}\mathrm{(mP)}}\xspace}

\def\Rmn           {\ensuremath{\mathcal{R}^{\chibmp}_{\YnS}}\xspace}

%% END localsymbols
%% =============================================================================
%%%%%%%%%%%%%%%%%%%%%%%%%%%%%%%%%%%%%%%%%%%%%%%%%%
%                                                %
%    BEGINNING OF TEXT                           %
%                                                %
%%%%%%%%%%%%%%%%%%%%%%%%%%%%%%%%%%%%%%%%%%%%%%%%%%
\begin{document}
%\linenumbers

%%%%%%%%%%%%%%%%%%%%%%%%%%%%%%%%%%%%%%%%%%%%%%%%%%%%%%%%%%%%%%%%%%
% The preamble of the paper
%%%%%%%%%%%%%%%%%%%%%%%%%%%%%%%%%%%%%%%%%%%%%%%%%%%%%%%%%%%%%%%%%%

\title{Study of \chib~meson
  spectroscopy and production
  in \proton\proton~collisions at \mbox{$\sqrt{s}=7$}~and~\mbox{$8\tev$}
   at \lhcb}

\author{Alexander Mazurov \email{alexander.mazurov@cern.ch},  on behalf of the LHCb collaboration
}

\affiliation{School of Physics and Astronomy,  University of Birmingham}

% You may repeat \author and \affiliation as many times as necessary!

\date{}
% Print it out!
\maketitle

%%%%%%%%%%%%%%%%%%%%%%%%%%%%%%%%%%%%%%%%%%%%%%%%%%%%%%%%%%%%%%%%%%
% The preamble of the paper
%%%%%%%%%%%%%%%%%%%%%%%%%%%%%%%%%%%%%%%%%%%%%%%%%%%%%%%%%%%%%%%%%%

\begin{abstract}
A study of \chib~meson spectroscopy and production at \lhcb is performed on proton-proton collision data,
corresponding to~3.0\invfb  of integrated luminosity 
collected at centre-of-mass energies  \sqs = 7 and 8 \tev.
The fraction of \YnS~mesons originating from \chib~decays is measured as a function of
the~\ups~transverse momentum
in the~rapidity range $2.0 < y^{\PUpsilon} < 4.5$.
The~radiative transition of the \chibThreeP~meson to \Y3S~is observed for the~first time.
%%{\color{red}Using assumptions about the mass~splitting for~\chibThreeP~multiplet,} 
The~\chiboneThreeP mass is determined to be
\begin{equation*}
    m_{\chiboneThreeP} = 10515.7^{+2.2}_{-3.9}\stat ^{+1.5}_{-2.1}\syst \mevcc
\end{equation*}
where the first uncertainty is statistical and the second is  systematic.
\end{abstract}
%% BEGIN Introduction
\section{Introduction}
\label{sec:introduction}

The comparison of experimental data for prompt production of \mbox{S-wave} quarkonia,
{\it{e.g.}}  \jpsi~or~\Y1S~mesons, with theory predictions requires
knowledge of feed-down contributions from P-wave quarkonia
states, \eg~radiative \decay{\chib}{\ups\g}~decays.
This contribution could significantly influence
the~interpretation of the~measured polarization of S-wave vector
quarkonia. In addition, measurements of the~relative production rates of P-wave to S-wave
quarkonia, as well as the~tensor-to-vector ratios, provide valuable
information on colour-octet matrix elements.


Depending on the~relative orientation of the~quark spins,
the~\chib~states can be either scalar, vector or tensor mesons, denoted by
$\Pchi_{\bquark \mathrm{J}}$ with total angular momentum $\mathrm{J}=0,1,2$.
%%
The~fractions of \YnS~decays
originating from \chibmp~decays,
where $\mathrm{n}$~and~$\mathrm{m}$ are radial quantum numbers of
the~bound states, are defined as
\begin{equation}
  \mathcal{R}^{\chibmp}_{\YnS} \equiv
  \dfrac
      {\upsigma  \left( \proton\proton \to \Pchi_{\bquark1}\mathrm{(mP)} \mathrm{X} \right)}
      {\upsigma  \left( \proton\proton \to \YnS    \mathrm{X} \right)}
      \times \BR_1
      + \dfrac
      {\upsigma  \left( \proton\proton \to \Pchi_{\bquark2}\mathrm{(mP)} \mathrm{X} \right)}
      {\upsigma  \left( \proton\proton \to \YnS    \mathrm{X} \right)}
      \times \BR_2
      \label{eq:r},
\end{equation}
%%
where $\BR_{1(2)}$ denotes the branching fraction 
for the~decay \decay{\Pchi_{\bquark1(2)}\mathrm{(mP)}}{\YnS \g}.
Possible contributions from \decay{\Pchi_{\bquark0}\mathrm{(mP)}}{\YnS \g}~decays 
are neglected because  of the~small branching fraction for the~corresponding
radiative decays.

A measurement of the fractions $\mathcal{R}^{\chibmp}_{\YnS}$ was
recently performed by LHCb using calorimetry photons~\cite{Aaij:1746553}.
This measurements supersede earlier \lhcb~measurements~\cite{LHCb-PAPER-2012-015,LHCb-CONF-2012-020}
by increasing the~statistical precision and exploiting more decay modes 
in the rapidity range  $2.0<y<4.5$.
%% END Introduction
%% BEGIN Results
\section{Results}
\label{sec:Conclusion}


The measured fractions \Rmn~\cite{Aaij:1746553} are presented in Fig.~\ref{fig:frac}.
The results are dominated by the statistical uncertainties, and show no dependence on the \proton\proton~collision energy.
A~measurement of the~$\mathcal{R}^{\chibthreep}_{\Y3S}$~fraction is performed for the first time.
The large value of this fraction impacts the~interpretation
of experimental data on \ups~production and polarization.
When data on \ups~production and polarization are compared with theory predictions,
as well as when different theory predictions are compared among themselves, it is often implicitly assumed
that the~fraction of \Y3S~mesons produced by feed~down from higher states is small.
The~large measured value of  $\mathcal{R}^{\chibthreep}_{\Y3S}$
indicates that these assumptions need to be revisited. 


\begin{figure*}[t!]
  \setlength{\unitlength}{1mm}
  \centering
  \begin{picture}(150,120)
    %% =======================================================================
    \put(0,0){
      %%\includegraphics*[width=75mm, height=60mm]{frac_ups3s.eps}
      \includegraphics*[width=75mm, height=60mm]{Fig-c.eps}
    }
    \put(2,25){\begin{sideways}$\mathcal{R}^{\chibthreep}_{\Y3S}~~~~~~~~\left[\%\right]$ \end{sideways}}
    \put(35,0){$p_{\mathrm{T}}^{\Y3S}$}
    \put(60,0){$\left[\!\gevc\right]$}
    
    \put(55,53){\scriptsize \textcolor{blue}{\sqs=7\tev}}
    \put(55,49){\scriptsize \textcolor{red}{\sqs=8\tev}}
    
    \put(15,54){\tiny \textcolor{blue}{\ding{111}} \textcolor{red}{\ding{110}}  $\chibThreeP \to \Y3S \g$}
      
    \put(15,40){\lhcb}
    %% =======================================================================
    \put(0,60){
      %%\includegraphics*[width=75mm, height=60mm]{frac_ups1s.eps}
      \includegraphics*[width=75mm, height=60mm]{Fig-a.eps}
    }
    \put(2,85){\begin{sideways}$\mathcal{R}^{\chibmp}_{\Y1S}~~~~~~~~\left[\%\right]$ \end{sideways}}
    \put(35,60){$p_{\mathrm{T}}^{\Y1S}$}
    \put(60,60){$\left[\!\gevc\right]$}
    
    \put(55,113){\scriptsize \textcolor{blue}{\sqs=7\tev}}
    \put(55,109){\scriptsize \textcolor{red}{\sqs=8\tev}}
    
    \put(14.8,114){\tiny \textcolor{blue}{\ding{109}} \textcolor{red}{\ding{108}} $\chibOneP \to \Y1S \g$}
    
    \put(14.8,111){\scalebox{0.6}{\textcolor{blue}{$\vartriangle$} \textcolor{red}{$\blacktriangle$}}\,\tiny$\chibTwoP \to \Y1S \g$}
    
    \put(15,108){\tiny \textcolor{blue}{\ding{111}} \textcolor{red}{\ding{110}} $\chibThreeP \to \Y1S \g$}    
    
    \put(15,100){\lhcb}

    %% =======================================================================
    \put(75,60){
      %%\includegraphics*[width=75mm, height=60mm]{frac_ups2s.eps}
      \includegraphics*[width=75mm, height=60mm]{Fig-b.eps}
    }
    \put(77,85){\begin{sideways}$\mathcal{R}^{\chibmp}_{\Y2S}~~~~~~~~\left[\%\right]$ \end{sideways}}
    \put(110,60){$p_{\mathrm{T}}^{\Y2S}$}
    \put(135,60){$\left[\!\gevc\right]$}
        
    \put(130,113){\scriptsize \textcolor{blue}{\sqs=7\tev}}
    \put(130,109){\scriptsize \textcolor{red}{\sqs=8\tev}}
    
    
    \put(89.8,114){\scalebox{0.6}{\textcolor{blue}{$\vartriangle$} \textcolor{red}{$\blacktriangle$}}\,\tiny$\chibTwoP \to \Y2S \g$}
    
    \put(90,111){\tiny \textcolor{blue}{\ding{111}} \textcolor{red}{\ding{110}} $\chibThreeP \to \Y2S \g$}  

    \put(90,100){\lhcb}

    % \graphpaper[5](0,0)(150, 120)
 \end{picture}
 \caption{\small
        Fractions \Rmn as functions of $p_{\mathrm{T}}^{\ups}$~\cite{Aaij:1746553}.
        Points with blue~open\,(red solid) symbols
        correspond to data collected at $\sqs=7(8)\,\mathrm{TeV}$, respectively.
        For better visualization the~data points are slightly displaced
        from the~bin centres.
        The~inner error bars represent statistical uncertainties,
        while the~outer error bars indicate statistical and systematic uncertainties added in quadrature. }
 \label{fig:frac}
\end{figure*}


The recent results supersede earlier \lhcb~measurements~\cite{LHCb-PAPER-2012-015,LHCb-CONF-2012-020}
by increasing the~statistical precision and exploiting more decay modes and higher transverse 
momentum regions.
In particular, the full data sample collected by \lhcb at \mbox{$\sqs=7$}~and 8\tev has been used and
the~measured fractions $\mathcal{R}^{\chibmp}_{\YnS}$
are reported for all six kinematically allowed transitions:
\mbox{\decay{\chibonep}{\Y1S\g}},
\mbox{\decay{\chibtwop}{\Y1S\g}},
\mbox{\decay{\chibtwop}{\Y2S\g}},
\mbox{\decay{\chibthreep}{\Y1S\g}},
\mbox{\decay{\chibthreep}{\Y2S\g}} and 
\mbox{\decay{\chibthreep}{\Y3S\g}}
in bins of transverse momentum of the~\ups~mesons in the rapidity range  $2.0<y<4.5$.
\lhcb also measured the fractions $\mathcal{R}^{\chibmp}_{\YnS}$ with the radiative photon reconstruted from a $\epem$ pair~$\mathcal{R}^{\chibmp}_{\YnS}$~\cite{Aaij:1753641}. The results are consistent with those obtained in~\cite{Aaij:1746553}.

Using photons converted into electron pairs, and assuming the mass splitting $m_{\chibtwoThreeP}-m_{\chiboneThreeP}=10.5\mevcc$,
the~mass of the \chiboneThreeP~state is measured to be 
\begin{equation*} 
m_{\chiboneThreeP} = 10515.7^{+2.2}_{-3.9}\stat ^{+1.5}_{-2.1}\syst \mevcc,
\end{equation*}
This result \cite{Aaij:1753641} is compatible and significantly more precise than 
the event yield average mass of
\chiboneThreeP~and
\chibtwoThreeP~states,
\mbox{$10\,530 \pm 5\pm 17\mevcc$} and 
\mbox{$10\,551 \pm 14\pm 17\mevcc$}, 
reported by the~\atlas~\cite{Aad:2011ih}
and D0~\cite{Abazov:2012gh} experiments, respectively.
%% END Results

%%%%%%%%%%%%%%%%%%%%%%%%%%%%%%%%%%%%%%%%%%%%%%%%%%%%%%%%%%%%%%%%%%
% References
%%%%%%%%%%%%%%%%%%%%%%%%%%%%%%%%%%%%%%%%%%%%%%%%%%%%%%%%%%%%%%%%%%
\begin{thebibliography}{99}
\bibitem{Aaij:1746553} LHCb collaboration, R. Aaij et al., \textit{Study of $\chi_{b}$ meson production in pp collisions at
                       $\sqrt{s}=7$ and 8TeV and observation of the decay
                       $\chi_{b}(3P) \rightarrow \Upsilon(3S) \gamma$}, Eur. Phys. J. C 74,3092 (2014).
\bibitem{LHCb-PAPER-2012-015} LHCb collaboration, R. Aaij et al., \textit{Measurement of the fraction of $\Upsilon(1S)$ originating from $\chi_b(1P)$ decays in $pp$ collisions at $\sqrt{s}=7$ TeV}, JHEP 11 (2012) 031
\bibitem{LHCb-CONF-2012-020} LHCb collaboration, \textit{Observation of $\chi_b(3P)$ state at LHCb in pp collisions at $\sqrt{s}=7\tev$}, LHCb-CONF-2012-020
\bibitem{Aaij:1753641}LHCb collaboration, R. Aaij et al., \textit{Measurement of the $\chi_b(3P)$ mass and of the relative
                       rate of $\chi_{b1}(1P)$ and $\chi_{b2}(1P)$ production}, JHEP 14, 88 (2014).
\bibitem{Aad:2011ih} ATLAS collaboration, G. Aad et al., \textit{Observation of a new $\chi_b$~state in radiative transitions to $\Upsilon(1S)$~and $\Upsilon(2S)$~at ATLAS}, Phys. Rev. Lett. 108 (2012) 152001
\bibitem{Abazov:2012gh} D0 collaboration, V. M. Abazov et al \textit{Observation of a narrow mass state decaying into $\Upsilon(1S) + \gamma$ in $p\bar{p}$~collisions at $\sqrt{s} = 1.96$ TeV}, Phys. Rev. D86 (2012) 031103.

\end{thebibliography}
\end{document}
